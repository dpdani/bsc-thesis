\chapter{Time Travel Functionality}
\label{ch:timetravel}

Every table has one representation in the source database and another representation in the destination database.
To avoid ambiguity, let us introduce the following notation:

\begin{itemize}
\item $\source{\omega}$ and $\dest{\omega}$ refer to the representation of some table $\omega$ at the source and destination databases respectively;
\item $\source{*}$ and $\dest{*}$ refer to the representations of all tables collectively (i.e. the whole databases), at the source and destination databases respectively.
\end{itemize}


We will assume the following, without proof:

\begin{axiom}
$\source{*}$ is inherently consistent.
\end{axiom}

The consistency of the source database should be assumed because a replication of a database can only be as consistent as the original database, i.e. there is no way to have consistent data, starting from inconsistent data.

Additionally, such is safe to assume, since the database constraints are checked at the source by the PostgreSQL server.

Generally, 
$\dest{\alpha} := \pi_{S}\left(\source{\alpha}\right)$
over some subset $S$ of its columns,
with one exception for table $\gamma$:

$$
\dest{\gamma} := \pi_{\dest{\alpha}.\text{id}, \source{\gamma}.*}
    \left(
        \source{\gamma} \bowtie_{\alpha.\text{email} = \gamma.\text{email}} \dest{\alpha}
    \right)
$$

\section{Empty Validity Ranges}

When dealing with ranges, one should account for the fact that $\emptyset$ is indeed a valid range:

$$
\left[t, t\right) = \emptyset, \forall t
$$

This has the inherent meaning, in our domain, that two different records for the same table were issued at the exact same time $t$.

At first, this may seem an issue that could possibly be dismissed as very unlikely.
Nevertheless, the presence of database transactions makes the significance of this scenario become evident.
Within a transaction, different queries are evaluated and later performed as a single atomic\footnote{??? explain atomic} operation, thus resulting in several records being issued at the same instant.

