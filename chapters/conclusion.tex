\chapter{Conclusion}
\label{ch:conclusion}

In this Thesis, the design and implementation of a particular CDC system have been presented, alongside the challenges encountered, the solutions adopted, and some ideas for possible future endeavors.

The presented system's development, which was carried out within the context of a curricular internship at SpazioDati, required the composite application of many topics that had been subject of studies during the Bachelor's course on Computer Science that I have attended; ranging from databases, to algorithms, and from functional as well as object oriented programming, to networking.

Throughout the Thesis chapters, different aspects have been covered with various levels of formality, according to what was deemed appropriate.
In certain cases, formal propositions and proofs have been attempted, e.g. such was the case for Proposition (\ref{prop:fk}).
In other cases, formality was deemed redundant and thus replaced with less precise assertions, see for instance the discussions in Chapter \ref{ch:future}.

Additionally, with regards to the system's implementation, discussion of source code was deemed relevant enough for it to be inserted in this Thesis, a selection of which can be found in Appendix \ref{ch:source}.
The choice of separating the source code from the rest of the document, was mainly a matter of avoiding cluttering discussions with listings.

At the time of writing, the system, as presented throughout this document, is actively running and processing data change events, within SpazioDati's infrastructure.

%I consider the work done during the aforementioned internship, as well as the drafting of this Thesis, to be a positive experience, and a stimulating exercise of the knowledge accumulated during the years spent studying at this institution.
