% Data flow diagram
% Author: David Fokkema
\documentclass{article}
\usepackage{tikz}
\usetikzlibrary{
	positioning,
	fit,
	arrows,
	petri,
	shapes.geometric,
	shapes.symbols
}

\usepackage[active,tightpage]{preview}
\setlength\PreviewBorder{0pt}%


\begin{document}
\begin{center}
\begin{preview}
\begin{tikzpicture}[
	node distance=5mm,
	shorten >=1pt,
	->
]

%\draw[step=5mm, black, thin] (-3,-5) grid (7,5);

\node[draw, cylinder, shape border rotate=90, aspect=0.1] (broker-1) {Broker};
\node[draw, cylinder, shape border rotate=90, aspect=0.1, below=of broker-1, xshift=2mm] (broker-2) {Broker};
\node[draw, cylinder, shape border rotate=90, aspect=0.1, right=of broker-1, yshift=-2mm] (broker-3) {Broker};
%\path (broker-1) edge [<->] (broker-2);
%\path (broker-2) edge [<->] (broker-3);
%\path (broker-3) edge [<->] (broker-1);
\begin{scope}
	\node[draw, fit=(broker-1)(broker-2)(broker-3), label=above:{Kafka Cluster}] (kafka) {};
\end{scope}


\node[left=of kafka, yshift=5mm] (prod) {Producers};
\node[draw, below=of prod] (prod-1) {\phantom{=}};
\node[draw, below=of prod-1] (prod-2) {\phantom{=}};
\node[draw, below=of prod-2] (prod-3) {\phantom{=}};
\node[below=of prod-3, yshift=5mm] {\vdots};


\node[right=of kafka, yshift=5mm] (cons) {Consumers};
\node[draw, below=of cons] (cons-1) {\phantom{=}};
\node[draw, below=of cons-1] (cons-2) {\phantom{=}};
\node[draw, below=of cons-2] (cons-3) {\phantom{=}};
\node[below=of cons-3, yshift=5mm] {\vdots};


\path (prod-1) edge [bend left] (broker-2);
\path (broker-2) edge [bend right] (cons-2);

\end{tikzpicture}
\end{preview}
\end{center}
\end{document}
