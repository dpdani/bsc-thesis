\chapter*{Abstract} % senza numerazione
\label{ch:abstract}

\addcontentsline{toc}{chapter}{Abstract} % da aggiungere comunque all'indice

%Sommario è un breve riassunto del lavoro svolto dove si descrive l'obiettivo, l'oggetto della tesi, le 
%metodologie e le tecniche usate, i dati elaborati e la spiegazione delle conclusioni alle quali siete arrivati.  
%
%Il sommario dell’elaborato consiste al massimo di 3 pagine e deve contenere le seguenti informazioni:
%\begin{itemize}
%  \item contesto e motivazioni 
%  \item breve riassunto del problema affrontato
%  \item tecniche utilizzate e/o sviluppate
%  \item risultati raggiunti, sottolineando il contributo personale del laureando/a
%\end{itemize}


This Theses contains the work I've carried out during the internship, part of the Bachelor's Degree in Computer Science curriculum, at SpazioDati Srl, a technology company based in Trento, Italy.

The work pertained the development of a Change Data Capture system.

My work will be based on a need for the company to move some part of their database to a new database, due to load constraints on the former database.
The transfer of data would be carried out using Kafka streams, linked to changes in the main database.

The new database would serve primarily their sales team, which would use it for usage analytics for existing customers.
This database will co-exist with the previous one, i.e. it will not replace the parts of it that it's replicating, but it would instead keep being updated with the change events triggering in the main database.


As the contents of this document pertains the work I've carried out within SpazioDati, all information relevant to their intellectual properties has been anonymized so as to not disclose it.
The methods for such anonymization have been portrayed in the following Chapter \ref{ch:non-disclosure}.
Nowhere in this document their customers' data is shown or cited, partly, or in whole.

In Appendix \ref{ch:source} the listings of source code are provided.
