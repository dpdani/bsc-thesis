\chapter*{Abstract} % senza numerazione
\label{ch:abstract}

\addcontentsline{toc}{chapter}{Abstract} % da aggiungere comunque all'indice

%Sommario è un breve riassunto del lavoro svolto dove si descrive l'obiettivo, l'oggetto della tesi, le 
%metodologie e le tecniche usate, i dati elaborati e la spiegazione delle conclusioni alle quali siete arrivati.  
%
%Il sommario dell’elaborato consiste al massimo di 3 pagine e deve contenere le seguenti informazioni:
%\begin{itemize}
%  \item contesto e motivazioni 
%  \item breve riassunto del problema affrontato
%  \item tecniche utilizzate e/o sviluppate
%  \item risultati raggiunti, sottolineando il contributo personale del laureando/a
%\end{itemize}


%This Thesis contains the work I've carried out during the internship, part of the Bachelor's Degree in Computer Science curriculum, at SpazioDati Srl, a technology company based in Trento, Italy.
%
%The work pertained the development of a Change Data Capture system.
%
%My work will be based on a need for the company to move some part of their database to a new database, due to load constraints on the former database.
%The transfer of data would be carried out using Kafka streams, linked to changes in the main database.
%
%The new database would serve primarily their sales team, which would use it for usage analytics for existing customers.
%This database will co-exist with the previous one, i.e. it will not replace the parts of it that it's replicating, but it would instead keep being updated with the change events triggering in the main database.


This Thesis presents work carried out during my curricular internship, which describes the design and implementation of a Change Data Capture system.
An interest of the marketing team gave rise to the demand for the presented system, as explained in Chapter \ref{ch:intro}, which in essence solves the problem of capturing data change events from a data source, sometimes aggregating such data, and storing them in a separate place for the marketing team to make further use thereof.

An introduction to the scope and rationale for the system is presented in the first chapter.
The several technologies that constitute the subsystems presented are explained in the second chapter, along with their structural composition into the overall system.
Next, several properties of the data at hand are introduced in Chapter \ref{ch:data} and one of them is explored in greater detail in Chapter \ref{ch:timetravel}.
Additionally, some paths for possible future enhancements are outlined in Chapter \ref{ch:future}.
In Appendix \ref{ch:source}, the listings of relevant source code are provided.

As the contents of this document to pertain the work I've carried out within SpazioDati, all information relevant to their intellectual property has been anonymized so as to not disclose it.
The methods for such anonymization are portrayed in \S \ref{sec:non-disclosure}.
Nowhere in this document is their customers' data shown or cited, partly, or in whole.
